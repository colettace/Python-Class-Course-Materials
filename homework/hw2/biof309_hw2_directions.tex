\documentclass[11pt]{amsart}
\usepackage{geometry}                % See geometry.pdf to learn the layout options. There are lots.
\geometry{letterpaper}                   % ... or a4paper or a5paper or ... 
%\geometry{landscape}                % Activate for for rotated page geometry
\usepackage[parfill]{parskip}    % Activate to begin paragraphs with an empty line rather than an indent
\usepackage{graphicx}
\usepackage{amssymb}
\usepackage{epstopdf}
\usepackage{pifont}
\usepackage{underscore}
\DeclareGraphicsRule{.tif}{png}{.png}{`convert #1 `dirname #1`/`basename #1 .tif`.png}

\title{Number guessing game}
\author{Homework 2, due Thursday, March 6, 2014}
%\date{}


\begin{document}
\maketitle

\section*{Files}
\begin{itemize}
  \item biof309_hw2_solution.pyc -  A byte compiled, fully playable solution to this assignment that is not human readable.
  \item biof309_hw2.py - A partially completed program where you fill in the blanks.
  \item biof309_test_hw2.py - Autograding program that will test your work.
  \item biof309_hw2_directions.pdf - This file.
\end{itemize}

\section*{Description of the Program}
The object of the game is for the user to guess an integer chosen randomly by the computer. At the start of the game the user chooses the range of values in which the random number should fall. The game can have up to ten
players who will take turns guessing. The program keeps track of the time it takes for each
player to guess their number and how many turns s/he needed.  At the end of the game it prints
out a report on how the contestants did, the winner being the one who took the least amount of
time to guess their number. The program then asks the user(s) if they want to play again.

\section*{Directions}
\begin{enumerate}
	\item Download the files. Run the solution code. Run the test code on the solution to see
        that it works.
	\item Rename the template file to follow the naming convention: yourlastname_firstinitial_hw2.py
	\item Fill in the blanks. You are not allowed to change the function signatures, meaning
        you can't change the name of the function you're filling in, nor can you change the
        arguments to those functions. It is also not allowed to change any other part of the
        program, including the other functions and their code. Play your game, and test that
        it emulates the behavior and functionality of the solution.
	\item Run the autograder code on your file on the command line with the command
        python biof309_test_hw2.py yourlastname_firstinitial_hw2.py to see how you did.
	\item Send me an email with your program as an attachment with the subject \\
        "\textless your last name\textgreater hw2 submission".
\end{enumerate}

\section*{Concepts used}
The concepts used in this program may include, but is not limited to, information capture (i.e., \texttt{input()} etc.), conditional statements (i.e., \texttt{if ... else} etc), dummyproofing (i.e., checking to see whether user's input is valid), loops, using an infinite loop (i.e., \texttt{while( True ):}, using \texttt{break} to get out), etc...

\section*{Python Packages to be Used}
\texttt{time}, \texttt{random}

\section*{Tips}
\begin{itemize}
	\item Each function has a docstring that says exactly what the function is supposed to do. Read
  the docstrings very carefully.
	\item You should read and understand the other parts of the program to see how the other parts
  will use your function, as well as for tips on how to structure your functions.
	\item This program should PEDANTICALLY check user input for validity. While you're writing your
  code, try to think about all the ways the user can enter the wrong thing, and write the code
  with guardrails to make sure it doesn't crash if the user enters garbage. At no point in
	playing the game should any SyntaxError or uncaught Exception be raised.
	\item While you're writing the code for your two functions, you might find it easier to test just
  the function you're working on independently, rather than run the whole program. You can do
	that by calling the functions individually from the "main" part of the program at the bottom
	of the template file, e.g., comment out the call to RunGame() and make the call to your function
  instead.
	\item When you run your code, you may observe that the program has a logic error (a bug) because
  it does not behave as intended. In order to track down bugs, you can insert print statements
	printing out the values of the variables you're using, or you can use the Python debugger
	to step through your code line by line to see what it's doing in real time.
  To do the latter, insert the line "import pdb; pdb.set_trace()" somewhere above the code
  where you suspect the bug is and run the program. The program will stop at the line after
	the import. Google around to learn about the commands used to drive the Python debugger.
\end{itemize}


\end{document}  